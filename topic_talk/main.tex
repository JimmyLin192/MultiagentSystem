%%%%%%%%%%%%%%%%%%%%%%%%%%%%%%%%%%%%%%%%%%%%%%%%%%%%%%%%%%%%%%%%%%%%%%%%
%%%  THIS TEX FILE IS TO GENERATE PDF FILE FOR 
%%% 
%%%  COPYRIGHT (C) JIMMY LIN, 2013, UT AUSTIN
%%%%%%%%%%%%%%%%%%%%%%%%%%%%%%%%%%%%%%%%%%%%%%%%%%%%%%%%%%%%%%%%%%%%%%%%
\documentclass[10pt]{beamer}
%%%%%%%%%%%%%%%%%%%%%%%%%%%%%%%%%%%%%%%%%%%%%%%%%%%%%%%%%%%%%%%%%%%%%%%%
%%%  PACKAGES USED IN THIS TEX SOURCE FILE
%%%%%%%%%%%%%%%%%%%%%%%%%%%%%%%%%%%%%%%%%%%%%%%%%%%%%%%%%%%%%%%%%%%%%%%%
\usepackage{graphicx}
\usepackage{wrapfig}
\usepackage{geometry}
\usepackage{color}
\usepackage{JSPPT}
\usepackage[style=ieee]{biblatex}
%%%%%%%%%%%%%%%%%%%%%%%%%%%%%%%%%%%%%%%%%%%%%%%%%%%%%%%%%%%%%%%%%%%%%%%%
%%% PRESENTATION INFORMATION
%%%%%%%%%%%%%%%%%%%%%%%%%%%%%%%%%%%%%%%%%%%%%%%%%%%%%%%%%%%%%%%%%%%%%%%%
\title{Multiagent Behaviors in Neural Network}
\author[Jimmy Lin and Barry Feigenbaum]{\bf Jimmy Lin \\Barry Feigenbaum}
\institute{\bf Prof. Risto Miikkulainen\\[0.3cm] Department of Computer Science \\The University of Texas At Austin}

%%%%%%%%%%%%%%%%%%%%%%%%%%%%%%%%%%%%%%%%%%%%%%%%%%%%%%%%%%%%%%%%%%%%%%%%
%%% TITLE PAGE
%%%%%%%%%%%%%%%%%%%%%%%%%%%%%%%%%%%%%%%%%%%%%%%%%%%%%%%%%%%%%%%%%%%%%%%%
\begin{document}
\begin{large}
    \frame{\maketitle}%\titlepage
\end{large}

\frame{
    \frametitle{Table of Contents}
    \tableofcontents
}

%%%%%%%%%%%%%%%%%%%%%%%%%%%%%%%%%%%%%%%%%%%%%%%%%%%%%%%%%%%%%%%%%%%%%%%%
%%% DOCUMENTATION STARTS FROM HERE
%%%%%%%%%%%%%%%%%%%%%%%%%%%%%%%%%%%%%%%%%%%%%%%%%%%%%%%%%%%%%%%%%%%%%%%%
\section{Overview of Multiagent System}
\setbeamertemplate{itemize item}[ball]
\setbeamertemplate{itemize subitem}{$\int$}
% recapture contents of previous topic talks
\frame{
    \frametitle{Recap}
        What we currently have? 
            \begin{itemize}
                \item Commitee Machine 
                \item Reinforcement Learning
                \item Neuro-evolution
                \item High-level Behaviors
            \end{itemize}
}
\frame{
    \frametitle{Motivation}
    \begin{itemize}
        \item Most of works so far has focused exclusively on single agents
            we can extend reinforcement learning straightforwardly to multiple
            agents if they are all independent. 
        \item Intuitive idea: Multiple agents together will outperform
            any single agent due to the fact that they have more resources and
            a better chance of receiving rewards.
        \item Today, we will touch a really broad area 
            $$ \text{"Multiagent System" (M.A.S.).}
            $$
    \end{itemize}
}
% introduce multiagent behavior
\frame{
    \frametitle{Introduction}
         What is multiple agent system?
            \begin{itemize}
                \item Unfortunately, it is not formally defined by M.A.S. community.
                \item Employment of multiple agents (10 to thousands). 
                \item Intelligent mechanisms to address interactions between
                    agents.
            \end{itemize}
         When is it proposed?
            \begin{itemize}
                \item a relatively new sub-field of computer science 
                \item has only been studied since about 1980
                \item only gained widespread recognition since
                    about the mid-1990s
            \end{itemize}
}
\frame{
    \frametitle{M.A.S. Environments}
    The agents in a multi-agent system have several important characteristics:
    \begin{itemize}
        \item Autonomy: the agents are at least partially independent, self-aware, autonomous.
        \item Local views: no agent has a full global view of the system, or
            the system is too complex for an agent to make practical use of
            such knowledge.
        \item Decentralization: there is no designated controlling agent (or
            the system is effectively reduced to a monolithic system).
    \end{itemize}
}
\frame{
    \frametitle{Applications/Simulations}
    \begin{itemize}
        \item
            \href{https://www.youtube.com/watch?v=pqBSNAOsMDc&index=12&list=PLD56A0C7765234DCD}{Crowd
                Simulation / Crowd Collision Avoidance}
        \item    \href{https://www.youtube.com/watch?v=Hc6kng5A8lQ&index=13&list=PLD56A0C7765234DCD}
            {ClearPath: Highly Parallel Collision Avoidance for Multi-agent Simulation}
        \item \href{https://www.youtube.com/watch?v=azIOCFjDZWA&list=PL5Tiw7DXCdu2roBlpHvUzOMvR7WGXMoP6}{MATISSE: A Multi-Agent based Traffic Simulation System}
    \end{itemize}
}
% 
\frame{
    \frametitle{Independent v.s. Cooperative Agents}
     Tang Ming (1993) \cite{1} studied the performance of cooperative agents,
     using independent agents as a benchmark. Here are the discoveries:
    \begin{itemize}
        \item Additional sensation from another agent is beneficial if it can
            be used efficiently.
        \item Sharing learned policies or episodes among agents speeds up
            learning at the cost of communication.
        \item For joint tasks, agents engaging in partnership can
            significantly outperform independent agents although they may
            learn slowly in the beginning.
    \end{itemize}
}

\frame{
    \frametitle{Main topics of M.A.S.}
    Currently active research areas of M.A.S. are advanced Multiagent
    Behaviors, as follows:
    \begin{itemize}
        \item Communication
        \item Cooperation and Coordination
        \item Negotiation
        \item Distributed Problem Solving
        \item Multi-agent Learning
        \item Fault Tolerance
    \end{itemize}
}
% explain each multiagent behavior
\frame{
    \frametitle{Communication}
    {\bf Communication} is defined as altering the state of the environment
    such that other agents can perceive the modification and decode
    information from it.
    \begin{itemize}
        \item Direct Communication
        \item Indirect Communication
    \end{itemize}
}
\frame{
    \frametitle{Cooperation and Coordination}
}
\frame{
    \frametitle{Negotiation}
}
\frame{
    \frametitle{Distributed Problem Solving}
    The state space of a large, joint multi-agent task can be overwhelming. An obvious way to tackle this is to use domain
knowledge to simplify the state space, often by providing a smaller set of
more “powerful” actions customized for the problem domain. 

An alternative has been to reduce complexity
by heuristically decomposing the problem, and hence the joint behavior, into
separate, simpler behaviors for the agents to learn. Such decomposition may be
done at various levels (decomposing team behaviors into sub-behaviors for each
agent; decomposing an agents’ behavior into sub-behaviors; etc.), and the
behaviors may be learned independently, iteratively (each depending on the
earlier one), or in a bottom-up fashion (learning simple behaviors, then
grouping into “complex” behaviors).
}
\frame{
    \frametitle{Multi-agent Learning}
}
\frame{
    \frametitle{Fault Tolerance}
}
%%%%%%%%%%%%%%%%%%%%%%%%%%%%%%%%%%%%%%%%%%%%%%%%%%%%%%%%%%%%%%%%%%%%%%%%
\section{Multiagent behaviors in Neural Network}
\frame{
    \frametitle{Reinforcement learning}
    \begin{itemize}
        \item Barry, here u go.
    \end{itemize}
}
%%%%%%%%%%%%%%%%%%%%%%%%%%%%%%%%%%%%%%%%%%%%%%%%%%%%%%%%%%%%%%%%%%%%%%%%
\section{Discussions}
\frame{
    \frametitle{Questions, Suggestions or Some Other Ideas?}
}

%%%%%%%%%%%%%%%%%%%%%%%%%%%%%%%%%%%%%%%%%%%%%%%%%%%%%%%%%%%%%%%%%%%%%%%%
\section{Our Research Project}
\frame{
    \frametitle{Our Research Project}
    \begin{itemize}
        \item Motivations
        \item Mechanisms
        \item Suggestions
        \end{itemize}
}

%%%%%%%%%%%%%%%%%%%%%%%%%%%%%%%%%%%%%%%%%%%%%%%%%%%%%%%%%%%%%%%%%%%%%%%%
%%% REFERENCES
%%%%%%%%%%%%%%%%%%%%%%%%%%%%%%%%%%%%%%%%%%%%%%%%%%%%%%%%%%%%%%%%%%%%%%%%
\section{Appendix}
\setbeamertemplate{bibliography item}[book]
\frame{
    \frametitle{Further Readings: Books}
    \begin{thebibliography}{9}
        \bibitem{wooldridge2009introduction} W. Michael. {\it An introduction to
            multiagent systems}. John Wiley \& Sons, 2009.
       \bibitem{shoham2008multiagent} S. Yoav, and K. L. Brown. 
       {\it Multiagent systems: Algorithmic, game-theoretic, and logical
           foundations}. Cambridge University Press, 2008.
       \bibitem{weiss1999multiagent} W, Gerhard, ed. {\it Multiagent systems:
           a modern approach to distributed artificial intelligence}. MIT press, 1999.
   \end{thebibliography}
}
\setbeamertemplate{bibliography item}[online]
\frame{
    \frametitle{Further Readings: Courses and Labs}
    \begin{thebibliography}{9}
        \bibitem{CS224Spr} Stanford CS224M: Multi Agent Systems (Spring 2013-14). 
       \href{http://web.stanford.edu/class/cs224m/}{HERE}

        \bibitem{CPSC689} MIT CPSC689: Special Topics in Multi-Agent Systems (Spring 2006). 
       \href{http://web.stanford.edu/class/cs224m/}{HERE}
        \bibitem{StanfordGroup} Stanford Multiagent Research Group. 
       \href{http://multiagent.stanford.edu/}{HERE}
       \bibitem{AARTLab} CMU Advanced Agent-Robotics Technology Lab. 
       \href{http://www.cs.cmu.edu/~softagents/multi.html}{HERE}

       \bibitem{MITLab} MIT Robust Open Multi-Agent Systems (ROMA) Research Group. 
       \href{http://ccs.mit.edu/roma/}{HERE}
    \end{thebibliography}
}
\setbeamertemplate{bibliography item}[text]
\frame{
    \frametitle{References}
    \begin{thebibliography}{9}
        \bibitem{tan1993multi} Tan, Ming. "Multi-agent reinforcement learning:
        Independent vs. cooperative agents." {\it Proceedings of the Tenth
            International Conference on Machine Learning}. Vol. 337. 1993.
    \end{thebibliography}
}
\end{document}
