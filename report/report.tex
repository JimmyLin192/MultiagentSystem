\documentclass[conference]{IEEEtran}
\usepackage[pdftex]{graphicx}
\usepackage{cite}

% correct bad hyphenation here
\hyphenation{op-tical net-works semi-conduc-tor}

\begin{document}
%
% paper title
% can use linebreaks \\ within to get better formatting as desired
% Do not put math or special symbols in the title.
\title{Multiagent Coordination in Roombas: From a Neural Network Perspective}


% author names and affiliations
% use a multiple column layout for up to three different
% affiliations
\author{\textbf{Jimmy Xin Lin} and \textbf{Barry Feigenbaum}}

% use for special paper notices
%\IEEEspecialpapernotice{(Invited Paper)}

% make the title area
\maketitle

% As a general rule, do not put math, special symbols or citations
% in the abstract
\begin{abstract}
The abstract goes here.
\end{abstract}

\IEEEpeerreviewmaketitle



\section{Introduction}

\section{Related Works}
\subsection{Coordinated Multiagent Reinforcement Learning}
\cite{nguyen2014decentralized} ..

\cite{zhang2011coordinated}

\cite{zhang2013coordinating}

\cite{banerjee2012sample}

\cite{kraemer2012informed}
\subsection{Coordinated Multiagent Neuroevolution}

\section{Problem Formulation}

\subsection{Structure of Roomba Environment}
Add description of Roomba module here...

\subsection{Multiagent Behaviors}

\subsection{Difficulties and Challenges}


\section{Implementation and Results}

\subsection{Reinforcement Learning}
Add implementation details of Q-Learning and variants here...

\subsection{Neuroevolution}
Add implementation details of Neuroevolution here...



\section{Results}

%\begin{figure}[!t]
%\centering
%\includegraphics[width=2.5in]{myfigure}
%\caption{Simulation Results.}
%\label{fig_sim}
%\end{figure}



\section{Conclusions}
The conclusion goes here. 

Future Works go here.

% trigger a \newpage just before the given reference
% number - used to balance the columns on the last page
% adjust value as needed - may need to be readjusted if
% the document is modified later
%\IEEEtriggeratref{8}
% The "triggered" command can be changed if desired:
%\IEEEtriggercmd{\enlargethispage{-5in}}

% references section

% can use a bibliography generated by BibTeX as a .bbl file
% BibTeX documentation can be easily obtained at:
% http://www.ctan.org/tex-archive/biblio/bibtex/contrib/doc/
% The IEEEtran BibTeX style support page is at:
% http://www.michaelshell.org/tex/ieeetran/bibtex/

\bibliographystyle{IEEEtran}
\bibliography{report}

% <OR> manually copy in the resultant .bbl file
% set second argument of \begin to the number of references
% (used to reserve space for the reference number labels box)

%\begin{thebibliography}{1}
%\bibitem{IEEEhowto:kopka}
%H.~Kopka and P.~W. Daly, \emph{A Guide to \LaTeX}, 3rd~ed.\hskip 1em plus
%  0.5em minus 0.4em\relax Harlow, England: Addison-Wesley, 1999.
%\end{thebibliography}




% that's all folks
\end{document}
